\pagestyle{fancy}

\fancyfoot[C]{\scshape \emph{Wybory prezydenckie 2020/2021}}
\fancyfoot[R]{Sekcja Techniczna}
\fancyfoot[L]{NKWD}

\unappendix

\begin{center}
  \LARGE{ \textsc{DLA UCZNIÓW}}
  \vspace{0.3cm}
  \hline
  \vspace{1.2cm}
  Wybory na prezydenta I LO w Krakowie
  \vspace{0.8cm}
\end{center}

\section{Głosowanie}
\normalsize
\normalshape

\hspace{0.5cm} Ze względu na stan pandemii Nowodworska Komisja Wyborów Demokratycznych, zdecydowała się na zorganizowanie zdalnych wyborów oraz uczyniła Sekcję Techniczną Samorządu Uczniowskiego jej realizatorem. \par
Głosowanie odbędzie się za pomocą platformy która roześle jednorazowe linki na maile wskazane przez uczniów zainteresowanych oddaniem głosu. Więcej informacji pojawi się podczas ciszy wyborczej, ale od razu zalecamy szukać w folderze spam.

\section{Dane}

\hspace{0.5cm} Z tego samego powodu wasi przewodniczący, jeżeli już tego nie zrobili, skontaktują się z wami z prośbą o podanie swoich danych do uczestnictwa w głosowaniu.\par
\vspace{0.3cm}
Uczniowie mają wybór, co do tego jakie dane podają. NKWD przyjmuje dwie możliwości:

%\vspace{0.3cm}
\begin{itemize}
  \item Prawidłowy\footnote{Zgodny z \href{https://tools.ietf.org/html/rfc3696}{RFC3696 sekcją 3, J. Klensin}} adres email.
\end{itemize}

\vspace{-0.3cm}
\hspace{1cm} lub
\vspace{-0.3cm}

\begin{itemize}
  %\itemsep-0.1cm
  \item Prawidłowy\footnote{Zgodny z \href{https://tools.ietf.org/html/rfc4880}{RFC4880, J. Callas}} klucz GPG w formacie ASCII i adres email z wcześniejszego punktu.
\end{itemize}
\vspace{0.3cm} %\par

%Druga opcja przwidziana jest dla zaawansowanych technicznie uczniów, którzy oczekują pełnej weryfikowalności oddanego głosu.\par
Odmowa przekazania danych skutkuje pozbawieniem ucznia czynnych praw wyborczych na okres wyborów prezydenckich w roku szkolnym 2020/2021.

\vfill
\begin{center}
\normalsize \textcolor{white!30!black}{CELOWO POZOSTAWIONE PUSTE}\end{center}
\vfill

\newpage

\fancyfoot[L]{Xe\LaTeX\ 3.14}
\vspace{3cm}
\appendix
\section*{}

\section{Kontakt}

Wszelkie pytania techniczne proszę przysyłać na:
\begin{center}\href{mailto:tech@nowodworek.rocks}{\texttt{tech@nowodworek.rocks}}\end{center}
Wszelkie pytania dotyczące ordynacji wyborczej proszę przysyłać na:
\begin{center}\href{mailto:nkwd@nowodworek.rocks}{\texttt{nkwd@nowodworek.rocks}}\end{center}

\section{Klucz walidacji}

Ten dokument podpisany jest kluczem GPG sekcji techicznej zanjdującym się pod:
\begin{center}\texttt{F0A11C0D9CB5F44A} na \texttt{hkp://keyserver.ubuntu.com}\\ \end{center}
\noindent Tym samym kluczem będą podpisane potwierdzenia głosów.

\section{Eksport z GPG}

Dla uczniów którzy zamierzają skorzystać z drugiej opcji, podajemy komende eksportu klucza publicznego z gpg w interesującym nas formacie:
\begin{center}\texttt{gpg2 --armor --export [nazwa klucza]}\end{center}

\vfill
\begin{center}
\LARGE \scshape semper in altum\end{center}
\vfill
