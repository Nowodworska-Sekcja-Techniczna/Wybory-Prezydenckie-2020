\section{Wynik całościowy}

Wynik wyborów wygląda następująco:

\begin{center}
  \pgfplotsset{width=7cm,compat=1.15}

  \begin{tikzpicture}
    \begin{axis}[
      title=Wynik,
      xbar,
      width=12cm, height=4.5cm, enlarge y limits=0.5,
      axis on top,
      xlabel={Głosy},
      yticklabels={Pocztowska Anna, Gawlik Filip, Nie oddane głosy},
      ytick=data,
      nodes near coords, nodes near coords align={horizontal},
      xmin=0,
    ]
    \addplot[fill=main!20!white] coordinates {(122,1) (66,2) (613, 3)};
    \end{axis}
  \end{tikzpicture}

\end{center}
Gratulujemy wygranej i życzymy miłego sprawowania urzędu!

\vfill
\begin{center}
  \LARGE \scshape \textcolor{white!30!black}{celowo pozostawione puste}
\end{center}
\vfill
