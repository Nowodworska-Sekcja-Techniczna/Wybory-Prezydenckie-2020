\section{Frekwencja}

\makeatletter
\newif\ifcircled@makesize
\tikzset{
  circled/.code={
    \tikzset{%
      /circled/.cd,
      #1
    }
  },
  circled defaults/.code={
    \tikzset{%
      circled={%
        base=green!30!black,
        other=gray!30,
        otherl=red!40!black,
        sep=2pt,
        width=2pt,
      }
    }
  },
  circled size/.style={%
    circled={%
      make size=true,
      size=#1,
    }
  },
  /circled/.cd,
  base/.store in=\circled@basecol,
  other/.store in=\circled@othercol,
  otherl/.store in=\circled@othercoll,
  sep/.store in=\circled@sep,
  width/.store in=\circled@width,
  size/.store in=\circled@size,
  make size/.is if=circled@makesize,
  make size=false,
  size=0pt,
  /tikz/circled defaults,
}
\newlength\charwidth
\newlength\chwidth
\newdimen\circled@cw
\newdimen\circled@cs
\newcommand*\circled[3][]{%
  \tikzset{%
    circled defaults,
    circled={#1}
  }%
  \circled@cw=\circled@width
  \ifcircled@makesize
    \circled@cs=\circled@size
    \setlength\charwidth{\circled@cs}%
    \addtolength{\charwidth}{.5\circled@cw}% half line width
    \tikz[baseline=(char.base)];
      \draw [line width=\circled@width, color=\circled@basecol] ([yshift=.5\charwidth]char.center) arc (90:90-#3*3.6:.5\charwidth) coordinate (b);
      \draw [line width=\circled@width, color=\circled@basecol] ([yshift=.5\charwidth]char.center) arc (90:90-#2*3.6:.5002\charwidth) coordinate (a);

      \draw [line width=\circled@width, color=\circled@othercol] (a) arc (90-#2*3.6:-270:.5001\charwidth);
      \draw [line width=\circled@width, color=\circled@othercoll] (b) arc (90-#3*3.6:-270:.5001\charwidth);
    }%
  \else
    \settowidth\charwidth{#2\,\%}%
    \settototalheight\chwidth{#2\,\%}%
    \ifdim\chwidth>\charwidth\let\charwidth\chwidth\fi
    \circled@cs=\circled@sep
    \addtolength{\charwidth}{2\circled@cs+.5\circled@cw}% twice inner sep plus half line width
    \tikz[baseline=(char.base)];
      \draw [line width=\circled@width, color=\circled@basecol] (char.north) arc (90:90-#2*3.6:.5\charwidth) coordinate (a);
      \draw [line width=\circled@width, color=\circled@othercol] (a) arc (90-#2*3.6:-270:.5\charwidth);
      \draw [line width=\circled@width, color=\circled@othercol] (b) arc (90-#3*3.6:90-#2*3.6:.5\charwidth);
    }%
  \fi
}
\makeatother

Tegoroczna frekwencja wyniosła:

\begin{center}
  \tikzset{circled size=4cm}
  \noindent\circled[width=4pt]{23.4}{73}

  \[
    \textcolor{green!30!black}{\blacksquare} \text{Zagłosowało} = 23.4\% \quad \textcolor{gray!30}{\blacksquare} \text{Nie zagłosowało} = 49.4\% \quad \textcolor{red!40!black}{\blacksquare} \text{Nie przekazało danych} = 27\%
  \]
\end{center}
