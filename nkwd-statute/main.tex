\pagestyle{fancy}
\documentclass[12pt]{article}
\usepackage[T1]{fontenc}
\usepackage[polish]{babel}
\usepackage[utf8]{inputenc}
\begin{document}

\section{Postanowienia Ogólne}
    \begin{center}
        \S 1\\
    \end{center}
    Ordynacja wyborcza określa zasady i tryb przeprowadzania wyborów na urząd Prezydenta Samorządu Uczniowskiego, zwanego dalej Prezydentem.\\
    \begin{center}
        \S 2\\
    \end{center}    
    Wybory na urząd Prezydenta są powszechne, równe, bezpośrednie i odbywają się w głosowaniu tajnym.\\
    \begin{center}
        \S 3\\
    \end{center} 
    Czynne prawo wyborcze przysługuje każdemu uczniowi I Liceum Ogólnokształcącego im. Bartłomieja Nowodworskiego w Krakowie.\\
    \begin{center}
        \S 4\\
    \end{center} 
    Bierne prawo wyborcze posiadają uczniowie klas drugich I Liceum Ogólnokształcącego im. Bartłomieja Nowodworskiego w Krakowie.\\
    \begin{center}
        \S 5\\
    \end{center} 
    Prezydent I Liceum Ogólnokształcącego im. Bartłomieja Nowodworskiego w Krakowie wybierany jest na roczną kadencję.\\
    \begin{center}
        \S 6\\
    \end{center} 
    Termin wyborów prezydenckich i ramowe daty prowadzenia kampanii wyborczej zawarte są w Harmonogramie Wyborów Prezydenckich 2020\\

\section{Nowodworska Komisja Wyborcza}
    \begin{center}
        \S 7\\
    \end{center} 
    Nowodworską Komisję Wyborczą powołuje ustępujący Prezydent i wyznacza jej Przewodniczącego.\\
    \begin{center}
        \S 8\\
    \end{center} 
    Nowodworska Komisja Wyborcza odpowiada za prawidłowy przebieg wyborów.\\
    \begin{center}
        \S 9\\
    \end{center} 
    Nowodworska Komisja Wyborcza składa się z min. 3 uczniów i uczennic szkoły, którzy nie pełnią funkcji w Zarządzie Samorządu i nie mogą być uczniami tej samej klasy.\\
    \begin{center}
        \S 10\\
    \end{center} 
    Przewodniczący Nowodworskiej Komisji Wyborczej czuwa nad działaniem pozostałych członków Komisji zgodnym z Regulaminem Wyborów Prezydenckich.\\
    \begin{center}
        \S 11\\
    \end{center} 
    Każdy z kandydatów ma prawo powołać jednego Rzecznika Sztabu. Rzecznik Sztabu jest reprezentantem sztabu wyborczego kandydata przed Nowodworską Komisją Wyborczą.  Rzecznikowi  Sztabu przysługuje prawo składania wniosków, skarg lub odwołań do Nowodworskiej Komisji Wyborczej.\\
    \begin{center}
        \S 12\\
    \end{center} 
    Nowodworska Komisja Wyborcza nie rozpatruje wniosków, skarg i odwołań składanych przez innych członków sztabu wyborczego.\\
    \begin{center}
        \S 13\\
    \end{center} 
    Do zadań Nowodworskiej Komisji Wyborczej należą:
    \begin{enumerate}
        \item Ogłoszenie zasad zgłoszenia kandydata na Prezydenta.
        \item Przyjęcie zgłoszeń od kandydatów
        \item Weryfikacja zgłoszeń i ogłoszenie nazwisk kandydatów;
        \item Poinformowanie uczniów szkoły o zasadach głosowania;
        \item Czuwanie nad przebiegiem kampanii wyborczej, przygotowanie wyborów
        \item Przeprowadzenie głosowania;
        \item Obliczenie głosów;
        \item Sporządzenie protokołów z głosowania i ogłoszenie ich wyników;
        \item Przyjęcie i rozpatrzenie ewentualnych skarg na przebieg wyborów.
    \end{enumerate}

\section{Zgłoszenie Kandydata}
    \begin{center}
        \S 14\\
    \end{center} 
    Kandydatury zgłaszane są przez kandydatów Nowodworskiej  Komisji Wyborczej nie później niż  na 2 dni przed ustalonym terminem kampanii wyborczej.\\
    \begin{center}
        \S 15\\
    \end{center} 
    Komisja Wyborcza ma obowiązek przekazać kandydatury Dyrektorowi najpóźniej na dwa dni przed rozpoczęciem kampanii wyborczej.\\
    \begin{center}
        \S 16\\
    \end{center} 
    Kandydaci zobowiązani są do przedstawienia Nowodworskiej Komisji Wyborczej listy podpisów elektronicznych z poparciem dla kandydata w liczbie 43 podpisów elektronicznych na dzień przed rozpoczęciem kampanii wyborczej w formie dokumentu elektronicznego. Podstawą do umieszczenia podpisu na liście poparcia jest uprzednie wysłanie wiadomości za pośrednictwem serwisu Librus przez ucznia I Liceum do opiekuna odpowiedniego kandydata o treści wskazującej na udzielone poparcie. Na listę poparcia nie mogą zostać wpisane osoby będące członkami sztabu wyborczego kandydata.\\
    \begin{center}
        \S 17\\
    \end{center} 
    Kandydaci zobowiązują się do przedstawienia Nowodworskiej Komisji Wyborczej  listy członków ich sztabów najpóźniej na dzień dni przed rozpoczęciem kampanii wyborczej.\\
    \begin{center}
        \S 18\\
    \end{center} 
    Każdy z kandydatów zobligowany jest do wybrania Opiekuna Sztabu Wyborczego i przedstawienia go  Nowodworskiej Komisji Wyborczej najpóźniej na dzień przed rozpoczęciem kampanii wyborczej.\\
    \begin{center}
        \S 19\\
    \end{center} 
    Kandydaci zobowiązują się dostarczyć Nowodworskiej Komisji Wyborczej następujące dokumenty : Zaświadczenie od wychowawcy (zał. 1), Oświadczenia  Rodzica  (zał.2)  oraz Zgodę Opiekuna Sztabu (zał. 3) najpóźniej na dwa dni przed rozpoczęciem kampanii wyborczej.\\
    \begin{center}
        \S 20\\
    \end{center} 
    Niedopełnienie paragrafów 14-20 niniejszego regulaminu skutkuje niedopuszczeniem kandydata do udziału w wyborach.\\

\section{Kampania Wyborcza}
    \begin{center}
        \S 21\\
    \end{center} 
    Kandydat musi prowadzić kampanię wyborczą w sposób etyczny, nienaruszający dóbr i dobrego imienia kontrkandydatów  i Szkoły.\\
    \begin{center}
        \S 22\\
    \end{center} 
    Naruszenie dobrego imienia kontrkandydatów skutkuje naganą lub odebraniem biernego prawa wyborczego.\\
    \begin{center}
        \S 23\\
    \end{center} 
    Z powodów epidemiologicznych kampania wyborcza prowadzona jest wyłącznie za pośrednictwem określonych mediów społecznościowych w internecie.\\
    \begin{center}
        \S 24\\
    \end{center} 
    \begin{enumerate}
        \item Plakaty:
            \begin{itemize}
                \item zostaną umieszczone w mediach społecznościowych szkoły przez Nowodworską Komisję Wyborczą
                \item muszą zostać dostarczone Nowodworskiej Komisji Wyborczej najpóźniej do pierwszego dnia trwania kampanii wyborczej w formie elektronicznej i w formacie obsługiwanym przez Instagram oraz Facebook
                \item każdy kandydat może udostępnić do dwóch plakatów.
            \end{itemize}
        \item Ulotki:
            \begin{itemize}
                \item zostaną umieszczone w mediach społecznościowych szkoły przez Nowodworską Komisję Wyborczą
                \item muszą zostać dostarczone Nowodworskiej Komisji Wyborczej najpóźniej do pierwszego dnia trwania kampanii wyborczej w formie elektronicznej i w formacie obsługiwanym przez Instagram oraz Facebook
                \item nie mogą zawierać informacji o wynegocjowanej zniżce bez wcześniejszego przedstawienia Komisji Wyborczej wypełnionego Oświadczenia o zniżce w lokalu. 
            \end{itemize}
        \item Spot wyborczy:
            \begin{itemize}
                \item jest opcjonalnym elementem kampanii wyborczej
    
                \item musi zostać dostarczony  Nowodworskiej Komisji Wyborczej w terminie określonym przez Komisję.
    
                \item nie może być dłuższy niż 5 minut
    
                \item musi zostać dostarczony  Nowodworskiej Komisji Wyborczej w postaci łącza internetowego do wirtualnego dysku umożliwiającego przechowywanie i synchronizację plików w formacie akceptowanym przez serwis YouTube
                
                \item zostanie umieszczony na kanale Nowodworek TV na YouTube oraz udostępniony w grupie Nowodworek Info na Facebooku
            \end{itemize}
    
    
        \item Biuletyn wyborczy:
            \begin{itemize}
                \item pytania do wywiadu zostaną podane Kandydatom w dzień rozpoczęcia kampanii wyborczej
                \item przesyłając materiał do Redaktora Naczelnego The Nowodworek Times, Kandydat wyraża zgodę na upublicznienie treści w Biuletynie Wyborczym i na przetwarzanie danych osobowych zgodnie z Rozporządzeniem Parlamentu Europejskiego i Rady (UE) 2016/679 z dnia 27 kwietnia 2016 r. w sprawie ochrony osób fizycznych w związku z przetwarzaniem danych osobowych i w sprawie swobodnego przepływu takich danych oraz uchylenia dyrektywy 95/46/WE (ogólne rozporządzenie o ochronie danych)
            \end{itemize}
        \item Spotkania w klasach:
            \begin{itemize}
                \item mogą odbyć się jedynie za dobrowolną zgodą nauczyciela prowadzącego lekcję w trybie zdalnym
                
                \item w spotkaniu, oprócz Kandydata, może uczestniczyć maksymalnie 2 członków sztabu wyborczego.
            \end{itemize}
        \item Wywiady z kandydatami:
            \begin{itemize}
                \item Każdy z kandydatów otrzyma zaproszenie do wzięcia udziału w wywiadzie przeprowadzonym przez twórców kanału Nowodworek TV na platformie Youtube
                \item Każdy z kandydatów otrzyma zaproszenie do wzięcia udziału w wywiadzie przeprowadzonym przez administratorów profilu NWD1158 na platformie Instagram. Podczas tego wywiadu kandydat będzie mógł także odpowiadać na pytania od uczniów szkoły.
            \end{itemize}
    
        \item Delegacje:
            \begin{itemize}
                \item są udzielane Kandydatom i członkom ich sztabów jedynie przez Opiekuna Kandydata, po zasięgnięciu opinii od Nowodworskiej Komisji Wyborczej.
            \end{itemize}
    \end{enumerate}
    \begin{center}
        \S 25\\
    \end{center} 
    Posty i relacje o charakterze ataku na kogokolwiek będą skutkować odebraniem czynnego lub biernego prawa wyborczego.\\
    \begin{center}
        \S 26\\
    \end{center} 
    W przypadku braku rozstrzygnięcia wyborów w I turze Nowodworska Komisja Wyborcza przeprowadza w sposób zdalny konferencję prasową z udziałem kandydatów, którzy awansowali do II tury wyborów. Na konferencji pytania zadaje publiczność. Konferencja jest nagrywana i udostępniana na kanale Nowodworek TV w serwisie internetowym YouTube. \\
    \begin{center}
        \S 27\\
    \end{center} 
    Wszystkie oficjalne materiały promujące kandydata (plakaty, ulotki i inne) muszą zostać  przedstawione Nowodworskiej Komisji Wyborczej.  Materiały zostają dopuszczone do publikacji po ich zatwierdzeniu przez Komisję.\\
    \begin{center}
        \S 28\\
    \end{center} 
    Cisza wyborcza:
    \begin{itemize}
        \item trwa  od godziny 00:00 dnia   poprzedzającego głosowanie   do ogłoszenia wyników głosowania przez Nowodworską Komisję Wyborczą,
        \item  cisza wyborcza przed II turą głosowania rozpoczyna się bezpośrednio  po zakończeniu konferencji prasowej,
        \item Złamanie ciszy wyborczej lub rozpoczęcie kampani przed jej oficjalnym startem  może skutkować odebraniem biernego prawa wyborczego. 
    \end{itemize}
    
\section{Debata Prezydencka}
    \begin{center}
        \S 29\\
    \end{center}
    Debata Prezydencka odbywa się w sposób zdalny  przy wykorzystaniu nowoczesnych metod komunikacji zdalnej.\\
    \begin{center}
        \S 30\\
    \end{center}
    W Debacie Prezydenckiej mogą wziąć udział uczniowie i nauczyciele I Liceum Ogólnokształcącego im. Bartłomieja Nowodworskiego w Krakowie oraz osoby zaproszone przez ustępujący Zarząd Samorządu Uczniowskiego.\\
    \begin{center}
        \S 31\\
    \end{center}
    Debatę rozpoczyna i prowadzi ustępujący Prezydent.\\
    \begin{center}
        \S 32\\
    \end{center}
     Nad porządkiem debaty czuwa Nowodworska Komisja Wyborcza.\\
    \begin{center}
        \S 33\\
    \end{center}
    Kandydaci losują zajmowane miejsca przed rozpoczęciem debaty w obecności członka Nowodworskiej Komisji Wyborczej.\\
    \begin{center}
        \S 34\\
    \end{center}
    Kolejność wypowiedzi kandydatów uzależniona jest od wylosowanego miejsca.\\
    \begin{center}
        \S 35\\
    \end{center}
    Głos przed rozpoczęciem debaty zabierają: Dyrektor I Liceum Ogólnokształcącego w Krakowie (lub osoba przez niego delegowana) oraz Przewodniczący Nowodworskiej Komisji Wyborczej.
    \begin{center}
        \S 36\\
    \end{center}
    Przebieg debaty
    \begin{enumerate}
        \item autoprezentacja kandydatów.
        \item pytania od Zarządu Samorządu Uczniowskiego oraz Nowodworskiej Komisji Wyborczej.
        \item pytania od przedstawicieli szkół K5
        \item pytania od kontrkandydatów.
        \item pytania od sztabów wyborczych
        \item pytania od publiczności
        \item Zakończenie debaty przez Prowadzącego.
    \end{enumerate}
    \begin{center}
        \S 37\\
    \end{center}
    Dokładny plan debaty, z uwzględnieniem czasu na odpowiedzi i ilości zadawanych pytań jest uzależniony od liczby kandydatów i zostanie przedstawiony Sztabom Wyborczym najpóźniej 7 dni przed wyznaczonym terminem debaty.\\
    \begin{center}
        \S 38\\
    \end{center}
    Debata będzie nagrywana i zostanie udostępniona na kanale Nowodworek TV w serwisie internetowym YouTube.
    
\section{Głosowanie}
    \begin{center}
        \S 39\\
    \end{center}
    Udział w głosowaniu jest dobrowolny.\\
    \begin{center}
        \S 40\\
    \end{center}
    Wybory odbywają się zdalnie z wykorzystaniem nowoczesnych narzędzi umożliwiających przeprowadzenie wyborów zgodnie z regulaminem.\\
    \begin{center}
        \S 41\\
    \end{center}
    Po przeprowadzeniu głosowania wyniki głosowania zdalnego zostają zapisane i udostępnione do publicznej wiadomości. \\
    \begin{center}
        \S 42\\
    \end{center}
    Oddany głos uznaje się za ważny, jeśli został oddany na jednego (zgłoszonego wcześniej) kandydata.\\
    \begin{center}
        \S 43\\
    \end{center}
    Po obliczeniu wszystkich głosów Nowodworska Komisja Wyborcza sporządza protokół z głosowania.\\
    \begin{center}
        \S 44\\
    \end{center}
    Przedstawiciele poszczególnych kandydatów (po jednym z każdego sztabu) mogą być obecni przy obliczaniu głosów.\\
    \begin{center}
        \S 45\\
    \end{center}
    Wybory kończą się na I turze, jeśli jeden z kandydatów uzyskał ponad połowę oddanych ważnych głosów. W przypadku nie uzyskania takiego wyniku odbywa się druga tura wyborów, której uczestnikami jest dwóch kandydatów, którzy w pierwszej  turze uzyskali dwa najwyższe wyniki.\\
    \begin{center}
        \S 46\\
    \end{center}
    Druga tura wyborów odbywa się, jeżeli w pierwszej turze wyborów żaden z kandydatów nie uzyska wymaganej większości głosów. II tura wyborów zostaje przeprowadzona do 3 dni roboczych od ogłoszenia wyników pierwszego głosowania.\\
    \begin{center}
        \S 47\\
    \end{center}
    Przepisy dotyczące przeprowadzania głosowania w II turze stosuje się odpowiednio.\\
    \begin{center}
        \S 48\\
    \end{center}
    Drugą turę wyborów wygrywa kandydat, który uzyskał zwykłą większość ważnie oddanych głosów.\\
    \begin{center}
        \S 49\\
    \end{center}
    Jeżeli do wyborów staje tylko jeden kandydat, przeprowadza się głosowanie, w którym musi wziąć udział minimum 70\% uczniów szkoły, aby zostało uznane za ważne. Kandydat zostaje uznany zwycięzcą wyborów jeśli w głosowaniu uzyska poparcie wyższe niż 50\% oddanych głosów ważnych. W przypadku braku osiągnięcia kworum głosowania jest powtarzane. W przypadku nieosiągnięcia przez kandydata poparcia wyższego niż 50\% oddanych głosów ważnych Nowodworska Komisja Wyborcza ogłasza nowe wybory prezydenckie.\\
    \begin{center}
        \S 50\\
    \end{center}
    Unieważnienie wyników głosowania:
    \begin{enumerate}
        \item jeżeli liczba głosów ważnych jest mniejsza niż 50\% wszystkich oddanych głosów, wówczas głosowanie uznaje się za nieważne;
        \item każde naruszenie przepisów zawartych w Rozdziale VIII Regulaminu Samorządu może być podstawą do unieważnienia wyników głosowania;
        \item o unieważnieniu wyników głosowania decyduje Dyrektor szkoły;
        \item w razie unieważnienia głosowania, do 10 dni roboczych od ogłoszenia wyników głosowania, rozpisuje się harmonogram ponownego głosowania;
    \end{enumerate}
    \begin{center}
        \S 51\\
    \end{center}
    Odwołanie od wyników głosowania Rzecznik Sztabu   może złożyć Nowodworskiej Komisji Wyborczej w ciągu 24 godzin od ogłoszenia wyników głosowania. Nowodworska Komisja Wyborcza zobowiązana jest do przedstawienia odwołania Dyrektorowi. 
    \begin{center}
        \S 52\\
    \end{center}
    Nowodworska Komisja Wyborcza ogłasza wyniki głosowania niezwłocznie po sporządzeniu Protokołu z głosowania. Wyniki głosowania stają się wynikami wyborów po upływie 24 godzin od ich ogłoszenia lub po zakończeniu procesu weryfikacji odwołań. 
    \begin{center}
        \S 53\\
    \end{center}
    Przekazanie urzędu Prezydentowi elektowi ma miejsce na zebraniu Zarządu Samorządu do 10 dni roboczych od daty ogłoszenia końcowych wyników wyborów.
\newpage

\section{Postanowienia Końcowe}
    \begin{center}
        \S 54\\
    \end{center}
    Nowodworska Komisja Wyborcza zastrzega sobie możliwość ingerencji w Regulamin Wyborów Prezydenckich I Liceum Ogólnokształcącego im. Bartłomieja Nowodworskiego w Krakowie na rok 2020, jeżeli wymagać tego będzie sytuacja epidemiologiczna związana z rozprzestrzenianiem wirusa SARS-CoV-2. Każda zmiana w Regulaminie wymaga zatwierdzenia przez Dyrektora I Liceum.\\
    \begin{center}
        \S 55\\
    \end{center}
    W przypadku rażącego złamania Regulaminu kandydat może być skreślony z listy osób kandydujących. 

\newpage
\appendix
\section*{}
\section{Terminarz wyborów}
Poniedziałek (23.11) - ostateczna data dostarczenia pierwszej części dokumentów, o których mowa w Regulaminie.\\
Wtorek (24.11) -  ostateczna data dostarczenia drugiej części dokumentów, o których mowa w Regulaminie.\\
Środa (25.11) - rozpoczęcie kampanii wyborczych\\
Piątek (04.12) - debata prezydencka\\
Niedziela (06.12) - cisza wyborcza przed pierwszą turą\\
Poniedziałek (07.12) - I Tura i ogłoszenie wyników głosowania\\
Wtorek (08.12) - Konferencja Prasowa. Po konferencji cisza wyborcza przed drugą turą\\
Środa (09.12) - II Tura i ogłoszenie wyników głosowania\\
\end{document}
